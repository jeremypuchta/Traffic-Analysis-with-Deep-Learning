\section{Ablauf}

Dieses Kapitel richtet den Blick auf den Ablauf des Systems. 
Dabei ist der Ablauf unterteilt in zwei unabhängige Phasen:

\begin{enumerate}
    \item Trainings- und Testingphase des neuronalen Netzes
    \item Betriebsphase der Webanwendung
\end{enumerate}

Zunächst wird der Ablauf der ersten Phase vorgestellt.
Diese befasst sich mit dem Training und Testing des neuronalen Netzes. 
Das verwendete neuronale Netz basiert auf der YOLO-Architektur, welche in [QUELLE] dokumentiert ist.
Weiterhin wird ein vortrainiertes Model zur Objekterkennung verwendet.
Dieses bietet bereits die Möglichkeit 80 Klassen zu erkennen und zu klassifizieren. 
Aufbauend auf diesem Fundament, wird das Model entsprechend der im Rahmen des Projekts aversierten Domäne optimiert.
Dabei spricht man von \textit{Training}.

Im Rahmen der Trainingsphase wird der zusammengestellte und mit Labeln versehene Datensatz verwendet, um 
das neuronale Netz zu trainieren. 
Hierbei werden die Bilder iterativ durch das neuronale Netz via \textit{Forward Pass}-Prozess [REFERENZ] geschickt.
Dabei detektiert und klassifiziert das neuronale Netz die im Bild vorliegenend Objekte und vergleicht anschließend
die Resultate mit den im Input-Datensatz manuell gelabelten Objekten.
Im Fehlerfall passt das neuronale Netz die Kantengewichte mit Hilfe der \textit{Back Propagation}-Technik [REFERENZ] an.
Zur Validierung des neuronalen Netzes werden dem Netz unbekannte Bilder als Input bereitgestellt.
Damit wird überprüft, ob das Netz in der Lage ist auf unbekannte Input-Daten zu reagieren und ob beispielsweise eine 
Überanpassung existiert.

Das Ergebnis der Trainingsphase ist ein auf den Projektanwendungsfall spezialisiertes Model.
Dieses Model wird im Folgenden in das Backend der Webanwendung integriert.
Zur Laufzeit der Webanwendung wird stündlich das aktuell veröffentlichte Webcambild (siehe \url{https://www.l.de/webcam.html}) gecrawled.
Dieses wird vom Flask-Backend durch das neuronale Netz geschickt. 
Das Resulat stellt die enthaltenen Objekte in einer JSON-Datei dar. 
Die enthaltenen relevanten Werte werden aus der JSON-Datei extrahiert und in eine Datenbank gespeichert.
Der gesammelte Datenbestand wird im Anschluss aufbereitet, harmonisiert und analysiert, bevor sie an das Frontend
geschickt werden.
Im Frontend werden die aus den Daten resultierenden Statistiken nutzerfreundlich visualisiert.
