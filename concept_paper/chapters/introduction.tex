\newpage
\section{Einleitung}

Die vorliegende Ausarbeitung dokumentiert die Konzeption des Projekts \textit{Traffic Analysis with Deep Learning}.

Das Ziel des Projekts ist die Erstellung einer Webanwendung, welche Daten über das 
Verkehrsaufkommen am Leipziger Ring sammelt, statistisch analysiert und visualisiert.
Als Datengrundlage werden die unter \url{https://www.l.de/webcam.html} öffentlich bereitgestellten Webcambilder
des Leipziger Rings verwendet. 
Diese werden regelmäßig mithilfe eines Webscraping-Algorithmus abgerufen und dem Datensatz hinzugefügt (\textbf{T1}). 

Innerhalb der gesammelten Bilddateien werden die für das aktuelle Verkehrsaufkommen relevanten Objekte erfasst.
Dazu wird ein für den Anwendungsfall der Objekterkennung vortrainiertes neuronales Netz implementiert.
Der Leistungsstand des neuronalen Netzes wird im ersten Schritt untersucht, bevor im zweiten Schritt eine
Verbesserung der Leistungsfähigkeit vorgenommen wird (\textbf{T2}). 
Anschließend erfolgt die Transformation der gesammelten Bilddaten in Zeitreihendaten (\textbf{T3}). 

Abschließend wird der Zeitreihendatenbestand für die nachfolgende Analyse und Visualisierung innerhalb der 
Webanwendung aufbereitet. 
Die Webanwendung stellt ein Dashboard zur Verfügung, welches die aufbereiteten Verkehrsdaten grafisch 
visualisiert (\textbf{T4}). 

Die aufbereiteten Verkehrsdaten geben Informationen über die gegenwärtige Verkehrslage am Leipziger Ring preis 
und ermöglichen es Rückschlüsse auf mögliche Über- oder Unterlastungen des betrachteten Verkehrsbereich zu ziehen.
Aus diesen Analysen ist es möglich Handlungen für eine Verbesserung der Verkehrssituation abzuleiten.  
Daher sind die Ergebnisse der Ausarbeitung von Interesse für politische Amtsträger, Studenten verschiedener 
Fachrichtungen sowie interessierte Bürger der Stadt Leipzig.